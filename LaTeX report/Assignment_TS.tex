\documentclass[10pt,a4paper]{article}
\usepackage[utf8]{inputenc}
\usepackage[T1]{fontenc}
\usepackage[english]{babel}
\usepackage{amsmath}
\usepackage{amsfonts}
\usepackage{amssymb}
\usepackage{graphicx}
\usepackage{mathpazo}
\author{ Antonio, Elena Proden, Hervégil Voegeli}
\title{Assignment for the course \\ \textit{The statistical analysis of time series}}
\begin{document}
	\maketitle
	\newpage
	\section{Theoretical aspects}
	\begin{enumerate}
	\item[[ 1.]] \begin{enumerate}
			\item[[ 1.1]] Compute the autocovariance function $\gamma_{X} (k) = \text{cov}(X_t,X_{t+k})$ for $k=0,1,\dots$ Is {\fontfamily{qzc} \selectfont \large  x } stationary? Do we need a restriction on $\theta$?
			
			\begin{align*}
				\gamma_{X}(k) &= \text{cov}(X_t,X_{t+k}) \\
				&= \text{E}((W_t+\frac{1}{\theta} W_{t-1})(W_{t+k}+\frac{1}{\theta}W_{t+k-1})) - \text{E}(W_t+\frac{1}{\theta} W_{t-1})\text{E}(W_{t+k}+\frac{1}{\theta}W_{t+k-1}) \\
				&= \text{E}((W_t+\frac{1}{\theta} W_{t-1})(W_{t+k}+\frac{1}{\theta}W_{t+k-1})) \\
				&= \text{E}(W_tW_{t+k} + \frac{1}{\theta} W_tW_{t+k-1}+\frac{1}{\theta} W_{t-1}W_{t+k} + \frac{1}{\theta^2} W_{t-1}W_{t+k-1}) \\
				&= 	\left\{
							\begin{array}{ll}
								\sigma^2_{W} (1+\frac{1}{\theta^2}) & h=0\\
								\frac{\sigma^2_{W}}{\theta} & |h|=1\\
								0 & \text{else}
							\end{array}
					\right.			
			\end{align*}		
			$\Rightarrow X_{t}$ is weakly stationary.\newline
			Apart from $\theta \neq 0$, to avoid $1+|\frac{1}{\theta}|  \rightarrow \infty$, no conditions for $\frac{1}{\theta}$ necessary to get stationarity on MA(q) processes.
			
			
			
			\item[[ 1.2]] Use the definition of Fourier transform of $\gamma_X$ and derive the expression of the spectral density, $f_X(\lambda)$, $\lambda \in \left[-1/2,1/2\right]$.
			
			\begin{align*}
				f_X(\lambda) &= \sum_{k=-\infty}^{\infty} \gamma_{X}(k) \text{exp}(-2\pi i \lambda k) \\
				&= \sum_{k=-2}^{-\infty} \gamma_{X}(k) \text{exp}(-2\pi i \lambda k) + \gamma_{X}(-1) \text{exp}(-2\pi i \lambda \cdot (-1)) \\
				&+ \gamma_{X}(0) \text{exp}(-2\pi i \lambda \cdot 0) + \gamma_{X}(1) \text{exp}(-2\pi i \lambda \cdot 1) + \sum_{k=2}^{\infty} \gamma_{X}(k) \text{exp}(-2\pi i \lambda k) \\
				&=  \sigma_{W}^2 \left( \frac{\theta (\text{exp}(2\pi i \lambda) + \text{exp}(-2\pi i \lambda) ) + \theta^2 + 1}{\theta^2}\right) \\
				&=\sigma_{W}^2 \left( \frac{\theta (\text{cos}(2\pi  \lambda) + i\text{sin}(2\pi  \lambda) + \text{cos}(2\pi \lambda) - i\text{sin}(2\pi \lambda)) + \theta^2 + 1}{\theta^2}\right) \\
				&= \frac{\sigma_{W}^2}{\theta^2}(2\theta \text{cos}(2 \pi \lambda)+\theta^2 +1)
			\end{align*}
			
			\end{enumerate}
	
	\item[[ 2.]]  \begin{enumerate}
		\item[[ 2.1]] Compute the autocovariance function $\lambda_Y(k) = \text{cov}(Y_t,Y_{t+k})$ for $k=0,1,\dots$
		\begin{align*}
			\gamma_{Y}(k) &= \text{cov}(Y_t, Y_{t+k}) \\
			&= \text{E}((\frac{1}{3}W_{t-1} + \frac{1}{3}W_t + \frac{1}{3}W_{t+1})((\frac{1}{3}W_{t+k-1} + \frac{1}{3}W_{t+k}  + \frac{1}{3}W_{t+k+1}))) \\
			&=  	\left\{
					\begin{array}{ll}
						\frac{\sigma_{W}^2}{3} & h=0\\
						\frac{2 \sigma_{W}^2}{9} & |h| = 1\\
						\frac{\sigma_{W}^2}{9} & |h| = 2 \\
						0 & \text{else}
					\end{array}
					\right.	
		\end{align*}
		
		\item[[ 2.2]] Use the definition of Fourier transform $\lambda_Y$ and derive the expression of the spectral density $f_Y(\lambda)$, $\lambda \in \left[-1/2,1/2\right]$
		\begin{align*}
			f_Y(\lambda) &= \sum_{k=-\infty}^{\infty} \gamma_{Y}(k) \text{exp}(-2\pi i \lambda k) \\
			&= \gamma_Y(-2) \text{exp}(4\pi i \lambda) + \gamma_Y(-1) \text{exp}(2\pi i \lambda) + \gamma_Y(0) \\
			&= + \gamma_Y(1) \text{exp}(\pi i \lambda) + \gamma_Y(2) \text{exp}(-4\pi i \lambda) \\
			&= \sigma_{W}^2 \left( \frac{\text{exp}(4\pi i \lambda) + \text{exp}(-4\pi i \lambda) + 2 \text{exp}(2\pi i \lambda) + 2 \text{exp}(-2\pi i \lambda) + 3 }{9}\right) \\
			&= \sigma_{W}^2 \left( \frac{2 \text{cos}(4\pi \lambda) + 4 \text{cos}(2\pi \lambda) + 3 }{9}\right)
		\end{align*}
		
		\end{enumerate}
	
	\item[[ 3.]] 
	\begin{enumerate}
		\item[[ 3.1]] Derive the expression for joint autocovariance function $\lambda_{WY}(k) = \text{cov}(W_{t+k},Y_t)$.
		\begin{align*}
			\gamma_{WY}(k) &= \text{cov}(W_{t+k},Y_t) \\
			&= \text{E}(W_{t+k}(\frac{1}{3}(W_{t-1}+W_{t}+W_{t+1}))) \\
			&= \frac{1}{3} \text{E}(W_{t+k}W_{t-1}+W_{t+k}W_{t}+W_{t+k}W_{t+1}) \\
			&=  	\left\{
			\begin{array}{ll}
				\frac{\sigma_{W}^2}{3} & h=0 \text{  or  } |h| = 1\\
				0 & \text{else}
			\end{array}
			\right.
		\end{align*}
		\item[[ 3.2]] Derive the expression of the cross-spectrum
		\begin{align*}
			f_{WY}(\lambda) &= \sum_{k=-\infty}^{\infty} \gamma_{WY}(k) \text{exp}(-2\pi i \lambda k) \\
			&= \gamma_{WY}(-1) \text{exp}(2\pi i \lambda) + \gamma_{WY}(0) \\ &= + \gamma_{WY}(1) \text{exp}(\pi i \lambda) \\
			&= \sigma_{W}^2 \left( \frac{\text{exp}(2\pi i \lambda) + \text{exp}(-2\pi i \lambda) + 1}{3}\right) \\
			&= \sigma_{W}^2 \left( \frac{\text{cos}(2\pi \lambda) + i\text{sin}(2\pi \lambda) +\text{cos}(-2\pi \lambda) + i\text{sin}(-2\pi \lambda) + 1}{3}\right) \\
			&= \sigma_{W}^2 \left( \frac{2\text{cos}(2\pi \lambda) + 1}{3}\right)
		\end{align*}
		\item[[ 3.3]] Let $f_W$ be the spectral density of the white noise process. Making use of the result in point 3.2, rearrange the terms and express $f_{WY}(\lambda)$ and provide the form of the term $|A_{WY}(\lambda)|^2$
		
		\begin{align*}
			f_W(\lambda) &= \sum_{k=-\infty}^{\infty} \gamma_{W}(k) \text{exp}(-2\pi i \lambda k) = \gamma_{W}(0) = \sigma_{W}^2\\
			f_{WY}(\lambda)&= \sigma_{W}^2 \left( \sum_{k=-1}^{1}(\frac{1}{\sqrt{3}}\text{exp}(2\pi i \lambda))^2\right) \\
			&= \sigma_{W}^2 |A_{WY}(\lambda)|^2 \\
			|A_{WY}(\lambda)|^2 &= \left( \sum_{k=-1}^{1}(\frac{1}{\sqrt{3}}\text{exp}(2\pi i \lambda))^2\right) \\
		\end{align*}
		\end{enumerate}
	\item[[ 4.]] Prove that $f_S(\lambda) = |A_{SQ}(\lambda)|^2 f_Q(\lambda)$ with $A_{SQ}(\lambda) = \sum_{j=-\infty}^{\infty}a_j \text{exp}(-2\pi i \lambda j)$
		\begin{enumerate}
		\item[[ 4.1]] Show that 
		\begin{align*}
			\gamma_{S}(k) &= \text{cov}(S_{t+k},S_t) \\
			&= \text{E}(\sum_{j=-\infty}^{\infty}a_j Q_{t+k-j} \cdot \sum_{j=-\infty}^{\infty}a_j Q_{t-j} ) \\
			&= \text{E} \left( (...a_-1 Q_{t+k+1} + a_0 Q_{t+k} + a_1 Q_{t+k-1} ...) \right. \\
			&\cdot (...a_-1 Q_{t+1} + a_0 Q_t + a_1 Q_{t-1} ...) \left. \right) \\
			&= ... \text{E}(a_-1 Q_{t+k+1} a_-1 Q_{t+1}) + \text{E}(a_-1 Q_{t+k+1} a_0 Q_t) \\
			&+ \text{E}(a_-1 Q_{t+k+1} a_1 Q_t-1)... \\
			&= \text{E}(\sum_{r} \sum_{s} a_s Q_{t+k-r} a_r Q_{t-s}) = \text{E}(\sum_{r} \sum_{s} a_r a_s Q_{t+k-r} Q_{t-s}) \\
			&= \sum_{r} \sum_{s} a_r a_s \text{E}(Q_{t+k-r} Q_{t-s}) = \sum_{r} \sum_{s} a_r a_s \gamma_Q (k-r+s) \\
			&= \sum_{r} \sum_{s} a_r a_s \left( \int_{-1/2}^{1/2} \text{exp}(2\pi i \lambda (k-r+s)) f_Q(\lambda)  \right) \\
		\end{align*}
		\item[[ 4.2]] Prove that
		\begin{align*}
			|A_SQ(\lambda)|^2 &= |\sum_{j=-\infty}^{\infty}a_j \text{exp}(-2\pi i \lambda j)|^2 \\
			&= \sum_{r} \sum_{s} a_r a_s  \text{exp}(-2\pi i \lambda r) \text{exp}(2\pi i \lambda s) \\
			&= \sum_{r} a_r \text{exp}(-2\pi i \lambda r) \sum_{s} a_s \text{exp}(2\pi i \lambda s) \\			
			\gamma_{S}(k) &= \sum_{r} \sum_{s} a_r a_s \left(\int_{-1/2}^{1/2} \text{exp}(2\pi i \lambda (k-r+s)) f_Q(\lambda) \right) \\			
			&= \sum_{r} \sum_{s} a_r a_s  \left(\int_{-1/2}^{1/2} \text{exp}(2\pi i \lambda k) \text{exp}(-2\pi i \lambda r) \text{exp}(2\pi i \lambda s) f_Q(\lambda) \right) \\			
			&= \int_{-1/2}^{1/2} \sum_{r} a_r \text{exp}(-2\pi i \lambda r) \sum_{s} a_s \text{exp}(2\pi i \lambda s) \text{exp}(2\pi i \lambda k) f_Q(\lambda) \\			
			&= \int_{-1/2}^{1/2} |A_SQ(\lambda)|^2 \text{exp}(2\pi i \lambda k) f_Q(\lambda)
		\end{align*}
	
		\begin{align*}			
			\text{By definition   } \gamma_{S}(k) &= \int_{-1/2}^{1/2} \text{exp}(2\pi i \lambda k) f_S(\lambda) d\lambda \\
			\text{Then   } \int_{-1/2}^{1/2} \text{exp}(2\pi i \lambda k) f_S(\lambda) d\lambda &= \int_{-1/2}^{1/2} |A_SQ(\lambda)|^2 \text{exp}(2\pi i \lambda k) f_Q(\lambda)\\
			f_S(\lambda) &= |A_SQ(\lambda)|^2 \frac{\text{exp}(2\pi \lambda k)}{\text{exp}(2\pi \lambda k} f_Q(\lambda) \\
			&|A_SQ(\lambda)|^2 f_Q(\lambda)
		\end{align*}
	\end{enumerate}
\end{enumerate}
	\section{Numerical exercises}
	Please see \textit{Assignment\_TS.rmd} for the analysis (code and comments).
\end{document}
